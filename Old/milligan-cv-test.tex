%%% A template to produce a nice-looking Curriculum Vitae.
%%% Kieran Healy <kjhealy@gmail.com>
%%% Most recent version is at http://kjhealy.github.com/kjh-vita
%%%
%%% ------------------------------------------------------------------------
%%% Requirements that are included in a modern tex distribution:
%%% ------------------------------------------------------------------------
%%% xelatex
%%% fontspec.sty
%%% hyperrref.sty
%%% xunicode.sty
%%% color.sty
%%% url.sty
%%% fancyhdr.sty
%%% memoir.cls
%%% fontawesome.sty
%%%
%%% 
%%% 
%%% ------------------------------------------------------------------------
%%% Requirements from https://github.com/kjhealy/latex-custom-kjh
%%% ------------------------------------------------------------------------
%%% org-preamble-xelatex.sty
%%% memoir-article-styles.sty
%%%
%%% ------------------------------------------------------------------------
%%% Optional
%%% ------------------------------------------------------------------------
%%% git
%%% vc.sty
%%% revnum.sty
%%% Fonts
%%%
%%% ------------------------------------------------------------------------
%%% Note
%%%------------------------------------------------------------------------
%%% Because this is a hand-tweaked file, be on the look out for \medksip, 
%%% \bigskip and \newpage commands here and there, which are used to balance
%%% the layout or avoid widows & orphans, etc. You should of course add or 
%%% remove these as needed.
%%%------------------------------------------------------------------------

\documentclass[11pt,article,oneside]{memoir}   
\usepackage{org-preamble-xelatex} 
\usepackage{fontawesome,url}


%%%------------------------------------------------------------------------
%%% Metadata
%%%------------------------------------------------------------------------

%% Change as needed. Or just add me as a coauthor. Only some of these are 
%% used below in the hyperref declaration and address banner section.
\def\myauthor{Ian Milligan}
\def\mytitle{Vita}
\def\mycopyright{\myauthor}
\def\mykeywords{}
\def\mybibliostyle{plain}
\def\mybibliocommand{}
\def\mysubtitle{}
\def\myaffiliation{University of Waterloo}
\def\myaddress{Department of History}
\def\myemail{i2millig@uwaterloo.ca}
\def\myweb{www.ianmilligan.ca}
\def\myphone{(519) 888-4567 x32775}
\def\myfax{(519) 746-2658}
\def\mytwitter{@ianmilligan1}
\def\myversion{}
\def\myrevision{}


\def\myaffiliation{University of Waterloo}
\def\myauthor{Ian Milligan}
\date{} % not used (revision control instead)
\def\mykeywords{Ian, Milligan, Ian Milligan, Vita, CV, Resume, History, Digital Humanities}

%%%------------------------------------------------------------------------  
%%% Git version tracking 
%%%------------------------------------------------------------------------

%% If you don't use git or the vc package (from CTAN), comment this out.
%% If you comment it out, be sure to remove the \rfoot comment below, too.
%\input{vc}

%%%------------------------------------------------------------------------
%%% Document
%%%------------------------------------------------------------------------
\begin{document}

%% Choose fonts for use with xelatex
%% Minion and Myriad are widely available, from Adobe. 
%% Pragmata is available to buy at http://www.fsd.it/fonts/pragma.htm
%% and is worth every penny. Any good monospace font will work fine, though.
%% Consolas or inconsolata are good alternatives.
\setromanfont[Mapping={tex-text}, 
	Numbers={OldStyle},
	Ligatures={Common},Scale=1.05]{Minion Pro}
\setsansfont[Mapping=tex-text,
	Ligatures={Common}, 
	Colour=AA0000,Scale=1.2]{Minion Pro}
\setmonofont[Mapping=tex-text,Scale=0.95]{Minion Pro} 

\newfontface\scheader[SmallCapsFont={Minion Pro},SmallCapsFeatures={Letters=SmallCaps}]{Minion Pro}

\newfontface\addressblock[Mapping={tex-text}, 
	Numbers={OldStyle},
	Ligatures={Common}]{Minion Pro}


%%%------------------------------------------------------------------------
%%% Local commands
%%%------------------------------------------------------------------------

%% Marginal header
%% Note: as the document goes on you may need to introduce a (gradually increasing)
%% \vspace element to keep the marginal header pleasingly aligned with the first 
%% item in the body text. Like this: \marginhead{{\vskip 0.4em}Grants}, or 
%% \marginhead{{\vskip 0.8em}Service}. Experiment as needed.
\newcommand{\marginhead}[1]{\marginpar{\textsf{{\footnotesize\vspace{-1em}\flushright #1}}}}


%% [optional] custom ampersand (font consistent with the one chosen above)
% \newcommand{\amper}{{\fontspec[Scale=.95,Colour=AA0000]{Minion Pro Medium}\selectfont\&\,}}

%% No bullets on labels
\renewcommand{\labelitemi}{~} 

%% Custom hanging indent for vita items
\def\ind{\hangindent=1 true cm\hangafter=1 \noindent}
%\def\ind{\hangindent=18pt\hangafter=1 \noindent}
\def\labelitemi{~}
\renewcommand{\labelitemii}{~}

%%%------------------------------------------------------------------------
%%% Page layout
%%%------------------------------------------------------------------------

% These lines will insert git revision info in the footer, using the vc
% package---see docs for vc package for details. Comment out this line
% if you're not using vc, and also remove the \input{vc} line above.
%\pagestyle{kjh}
%\thispagestyle{kjhgit}


%%%------------------------------------------------------------------------
%%% Address and contact block
%%%------------------------------------------------------------------------
\begin{minipage}[t]{2.95in}
 \flushright {\footnotesize 
 \href{http://uwaterloo.ca/history}{Department of History} \\ Hagey Hall of the Humanities 114 \\ University of Waterloo \\ \vspace{-0.05in} Waterloo ON Canada, N2L 3G1}  
  
\end{minipage}
\hfill     
%\begin{minipage}[t]{0.0in}
% dummy (needed here)
%\end{minipage}
\hfill
\begin{minipage}[t]{1.3in}
  \flushright \footnotesize  \addressblock \myphone \, \faPhone \\ 
  {\texttt{\href{http://twitter.com/ianmilligan1}{\mytwitter}} \, \faTwitter }  \\ 
  {\texttt{\href{mailto:\myemail}{\myemail}} \, \faEnvelope} \\
  {\texttt{\href{\myweb}{\myweb}} \, \faGlobe}
\end{minipage}

\medskip

%% Name 
\noindent{\LARGE\scheader \textsc{ian milligan}}
\reversemarginpar

\bigskip       


%% Appointments
% \medskip
\marginhead{\sffamily {{\vskip -0.23em} appointments}}

\ind Assistant Professor of History, University of Waterloo, 2012--Present.      

\ind Marshall McLuhan Centenary Fellow, Faculty of Information Studies, University of Toronto, 2016--17.

\ind Social Sciences and Humanities Council of Canada Postdoctoral Fellow, Western University, 2012.

\bigskip

%% Education

\marginhead{\sffamily {{\vskip -0.23em} education}}

%\noindent\emph{Princeton University \vspace{0.01in}}

\ind Ph.D, History, York University, 2012.

\ind M.A., History, York University, 2007. 

\ind B.A. (Hons.), History, Queen's University, 2006.

\bigskip
 
%% Publications
\marginhead{\sffamily {\vskip 0.35em} publications}
\medskip
\noindent\emph{Books \vspace{0.01in}}

\ind  Shawn Graham, Ian Milligan, and Scott Weingart, \emph{\href{http://www.worldscientific.com/worldscibooks/10.1142/p981}{Exploring Big Historical Data: The Historian's Macroscope}}. London:~Imperial College Press, 2015. \vspace{0.05in}

\ind  Ian Milligan, \emph{\href{http://www.ubcpress.ca/search/title_book.asp?BookID=299174315}{Rebel Youth: 1960s Labour Unrest, Young Workers, and New Leftists in English Canada}}. Vancouver:~University of British Columbia Press, 2014.\vspace{0.05in}

\ind \hspace{0.35in} \footnotesize Shortlisted, 2015 Sir John A. Macdonald Prize, Canadian Historical Association.

\vspace{-0.075in}

\normalsize

\bigskip
\noindent\emph{Journal articles \vspace{0.05in}}
 
%% Use revnumerate environment if numbered publications are needed. 
%% (Include it above in the preamble).
%% \renewcommand{\labelenumi}{\textsc{a}\theenumi.}
%% \begin{revnumerate}

\ind Jimmy Lin, Ian Milligan, Jeremy Wiebe, and Alice Zhou. ``Warcbase: Scalable Analytics Infrastructure for Exploring Web Archives,'' \emph{ACM Journal of Computing and Cultural Heritage}. Accepted and forthcoming, 2017.

\ind Nick Ruest and Ian Milligan. ``\href{http://journal.code4lib.org/articles/11358}{An Open-Source Strategy for Documenting Events: The Case Study of the 42nd Canadian Federal Election on Twitter}.'' \emph{Code4Lib Journal}, Issue 32, April 2016.

\ind Ian Milligan, Nick Ruest, and Anna St-Onge. ``\href{http://www.digitalstudies.org/ojs/index.php/digital_studies/article/view/325/412}{The Great WARC Adventure : Using SIPS, AIPS and DIPS to document SLAAPs}.'' \emph{Digital Studies/Le champ num\'erique}, Vol. 6, 2016.

\ind Ian Milligan. ``\href{http://www.euppublishing.com/doi/abs/10.3366/ijhac.2016.0161}{Lost in the Infinite Archive: The Promise and Pitfalls of Web Archives}.'' \emph{International Journal of Humanities and Arts Computing}, Vol. 10, No. 1-2 (2016):~87---94.

\ind Ian Milligan. ``\href{http://muse.jhu.edu/login?auth=0&type=summary&url=/journals/histoire_sociale_social_history/v048/48.96.milligan.pdf}{`A Haven for Perverts, Criminals, and Goons': Children and the Battle for and Against Canadian Internet Regulation, 1991-1999}.'' \emph{Histoire Sociale/Social History}, vol. 47, No. 96 (May 2015):~245---274.

\ind Ian Milligan. ``\href{http://muse.jhu.edu/journals/canadian_historical_review/toc/can.94.4.html}{Illusionary Order: Online Databases, Optical Character Recognition, and Canadian History, 1997-2010}.'' \emph{Canadian Historical Review}, Vol. 94, No. 4 (December 2013):~540---569.

\ind Ian Milligan. ``\href{http://www.erudit.org/revue/jcha/2012/v23/n2/1015788ar.html}{Mining the Internet Graveyard: Rethinking the Historians' Toolkit}.'' \emph{Journal of the Canadian Historical Association}, Vol. 23, No 2 (2012, published in 2013):~21---64. \textit{Winner, 2013 Journal of the Canadian Historical Association Best Article Prize}.

\ind Ian Milligan. ``\href{http://ojs.library.ubc.ca/index.php/bcstudies/article/view/2046}{Coming off the Mountain: Forging an Outward Looking New Left at Simon Fraser University}.'' \emph{BC Studies}, Vol. 171 (Autumn 2011):~69---91.

\ind Ian Milligan. ``\href{https://ianmilligan.ca/2013/06/17/post-firewall-this-board-has-a-duty-to-intervene/}{`This Board Has a Duty to Intervene,' Challenging the Spadina Expressway Through the Ontario Municipal Board, 1963-1971}.'' \emph{Urban History Review/Revue d'histoire urbaine}, Vol. 39, No. 2 (Spring 2011):~25---37.

\ind Ian Milligan. ``\href{http://www.lltjournal.ca/index.php/llt/article/download/5613/6476}{`The Force of All Our Numbers:'' New Leftists, Labour, and the 1973 Artistic Woodwork Strike}.'' \emph{Labour/Le Travail}, Vol. 66 (Fall 2010):~37---71.

\ind Ian Milligan. ``\href{http://search.proquest.com.proxy.lib.uwaterloo.ca/docview/208529434?accountid=14906}{Sedition in Wartime Ontario: The Trials and Imprisonment of Isaac Bainbridge, 1917-1918}.'' \emph{Ontario History}, Vol. 66 (Autumn 2008):~150---177.

%\end{revnumerate}
%\newpage
\bigskip

\noindent\emph{Book chapters \vspace{0.05in}}
% \renewcommand{\labelenumi}{\textsc{c}\theenumi.}
% \begin{revnumerate}

\ind Ian Milligan. ``Welcome to the Web: The Online Community of GeoCities and the Early Years of the World Wide Web,'' in \emph{The Web as History}, edited by Ralph Schroeder and Niels Br\"ugger. London: UCL Press, Forthcoming March 2017.

\ind William J. Turkel and Ian Milligan. ``The Challenge of `High-Throughput' Computational Methods,'' in \emph{Education and Understanding: Big History Around the World}, edited by Barry Rodrigue, Leonid Grinin, and Andrey Korotayev. New Delhi: Primus Books, 2016.

%\end{revnumerate}

\bigskip 

\noindent\emph{Computer Science conference publications \vspace{0.05in}}
% \renewcommand{\labelenumi}{\textsc{c}\theenumi.}
% \begin{revnumerate}

\ind Emily Maemura, Christoph Becker, and Ian Milligan. ``Understanding Computational Web Archives Research Methods Using Research Objects,'' \emph{IEEE Big Data Workshop in Computational Archival Science}, IEEE Big Data, December 2016. 

\ind Ian Milligan, Nick Ruest, and Jimmy Lin. ``\href{http://dl.acm.org/citation.cfm?id=2910913&CFID=806438388&CFTOKEN=32226772}{Content Selection and Curation for Web Archiving: The Gatekeepers vs. the Masses}.'' \emph{Proceedings of the ACM/IEEE Joint Conference on Digital Libraries}, Vol. 16 (2016):~107---110.

\ind Andrew Jackson, Jimmy Lin, Ian Milligan, and Nick Ruest. ``\href{http://dl.acm.org/citation.cfm?id=2910912&CFID=806438388&CFTOKEN=32226772}{Desiderata for Exploratory Search Interfaces to Web Archives in Support of Scholarly Activities}.'' \emph{Proceedings of the ACM/IEEE Joint Conference on Digital Libraries}, Vol. 16 (2016):~103---106.

%\end{revnumerate}

\bigskip 

\noindent\emph{Technical Contributions \vspace{0.05in}}
% \renewcommand{\labelenumi}{\textsc{c}\theenumi.}
% \begin{revnumerate}

\ind Ian Milligan. ``\href{http://www.ieee-tcdl.org/Bulletin/v11n2/papers/milligan.pdf}{Finding Community in the Ruins of GeoCities: Distantly Reading a Web Archive}.'' \emph{Bulletin of IEEE Technical Committee on Digital Libraries}, Vol. 11, Issue. 2  (October 2015).

\ind Ian Milligan and James Baker. ``\href{http://programminghistorian.org/lessons/intro-to-bash}{Introduction to the Bash Command Line}.'' \emph{Programming Historian}, September 2014.

\ind James Baker and Ian Milligan. ``\href{http://programminghistorian.org/lessons/research-data-with-unix}{Counting and Mining Research Data with Unix}.'' \emph{Programming Historian}, September 2014.

\ind Shawn Graham, Scott Weingart, and Ian Milligan. ``\href{http://programminghistorian.org/lessons/topic-modeling-and-mallet}{Getting Started with Topic Modeling and MALLET}.'' \emph{The Programming Historian}, September 2012. 

\ind Ian Milligan, ``\href{http://programminghistorian.org/lessons/automated-downloading-with-wget}{Automated Downloading with Wget},'' \emph{The Programming Historian}, August 2012. 

%\end{revnumerate}

\bigskip 

\noindent\emph{Essays \vspace{0.05in}}
% \renewcommand{\labelenumi}{\textsc{c}\theenumi.}
% \begin{revnumerate}

\ind Ian Milligan, ``\href{http://www.chronicle.com/article/The-Problem-of-History-in-the/238600?key=tZ1LWdA_bVNIm62Zk-gE3HKlSrriqn6ZRqJ3qEr6mmtHr9z8eisDICSwzJhXAzpuMlp1d01xalVFZE41SVk0cmFfeTh2U0lRYTBOazNSQlYxQklNR1BzYzdRbw}{The Problem of History in the Age of Abundance},'' \emph{The Chronicle of Higher Education}, December 2016.

\ind Ian Milligan, ``\href{http://www.revparl.ca/37/4/37n4e_14_Milligan.pdf}{Open Data's Potential for Political History},'' \emph{Canadian Parliamentary Review}, vol. 37, no. 4 (2014).

\ind Shawn Graham, Ian Milligan, and Scott Weingart, ``\href{http://www.historians.org/publications-and-directories/perspectives-on-history/october-2014/writing-the-historian\%E2\%80\%99s-macroscope-in-public}{Writing The Historian's Macroscope in Public},'' \emph{Perspectives on History: The Newsmagazine of the American Historical Association}, October 2014.

\ind Ian Milligan, ``\href{http://reviewcanada.ca/magazine/2014/05/does-history-matter/}{Does History Matter? Pioneering Research on Canada's Attitudes Towards Bygone Days},'' \emph{Literary Review of Canada}, May 2014:~28---29.

\ind Ian Milligan, ``Exploring the Web Through the Programming Historian,'' \emph{Canadian Historical Association Bulletin}, October 2013.

\ind Ian Milligan, ``\href{http://www.nature.com/nature/journal/v497/n7449/full/497317b.html}{Archives: Preserve Our Digital Heritage},'' \emph{Nature}, 497 (16 May 2013). 

%\end{revnumerate}
\bigskip 

%\newpage
\noindent\emph{Reviews \vspace{0.05in}}

%\renewcommand{\labelenumi}{\textsc{r}\theenumi.}
%\begin{revnumerate}

\ind Ian Milligan, ``Review of Lara Campbell, Dominique Cl\'ement, and Gregory S. Kealey, eds., Debating Dissent: Canada and the Sixties (Toronto: University of Toronto Press, 2012)'', in \emph{Canadian Historical Review}, Vol. 94, No. 4 (December 2013).

\ind Ian Milligan, ``Review of Benjamin Isitt, Militant Minority: British Columbia Workers and the Rise of a New Left, 1948-1972 (Toronto: University of Toronto Press, 2011)'', in \emph{Histoire Sociale/Social History}, Vol. 46, No. 92 (November 2013).

\ind Shawn Graham and Ian Milligan, ``\href{http://journalofdigitalhumanities.org/2-1/review-mallet-by-ian-milligan-and-shawn-graham/}{Review: MAchine Learning for LanguagE Toolkit (MALLET),}'' in \emph{Journal of Digital Humanities}, Vol. 2, No. 1 (Spring 2013).

\ind Ian Milligan, ``Review: ``\href{http://www.oralhistoryforum.ca/index.php/ohf/article/view/467/545}The Alberta Federation of Labour's Centennial Website: A Fantastic Resource (if you can find it),'' in \emph{Oral History Forum/Forum d'histoire orale} Vol. 33 (2013).

\ind Ian Milligan, ``Review of Jefferson Cowie, Stayin' Alive: The 1970s and the Last Days of the Working Class (New York: New Press, 2010)'', in \emph{Labour/Le Travail}, Vol. 69 (Spring 2012):~235---237.

\ind Ian Milligan, ``Review of Paul Mason, Live Working or Die Fighting: How the Working Class Went Global (Chicago: Haymarket Books, 2010)'', in \emph{Labor History}, Vol. 52, No. 3 (September 2011):~361---363.

\ind Ian Milligan, ``Review of James Pitsula, New World Dawning: The Sixties at Regina Campus (Regina: Canadian Plains Research Centre, 2008)'', in \emph{Canadian Historical Review}, 90:4 (December 2009):~818---820.

\ind Ian Milligan, ``\href{http://www.h-net.org/reviews/showrev.php?id=24198}{Review of Ian McKay, Reasoning Otherwise: Leftists and the People's Enlightenment in Canada, 1890-1920 (Toronto: Between the Lines, 2008)}'', \emph{H-Canada}, February 2009.

\bigskip 

\noindent\emph{Under Review\vspace{0.05in}}

\ind Ian Milligan, ``The Challenge and Potential of Big Data in History: Making Sense of Abundance,'' for \emph{Wiley Companion in Digital History}, ed. David Staley. Submitted November 2015. Contract signed and under review.

\ind Ian Milligan, ``Learning to `See' the Past at Scale: Exploring Web Archives through Hundreds of Thousands of Images,'' in \emph{Seeing the Past}, ed. Kevin Kee. Submitted December 2015. Under review.

\bigskip

\noindent\emph{Works Under Contract \vspace{0.05in}}

\ind Niels Br\"ugger, Megan Ankerson, and Ian Milligan. \emph{SAGE Handbook of Web History}. Under contract. Anticipated 2018.

\bigskip

%% Media

\marginhead{\sffamily {\vskip 0.5em}Media}
\medskip

\ind ``\href{http://www.cbc.ca/radio/thecurrent/the-current-for-november-5-2015-1.3305130/preserving-digital-history-is-imperative-to-save-cultural-history-1.3305263}{Preserving Digital History is Imperative to Save Cultural History},'' \emph{CBC's The Current with Anna Maria Tremonti}, 5 November 2015.

\ind ``\href{https://www.newscientist.com/article/mg22830422-900-how-to-save-our-digital-knowledge-for-future-generations-to-read/?utm_source=NSNS}{How to save our digital knowledge for future generations to read},'' \emph{New Scientist}, 7 October 2015.

\ind ``\href{http://www.cbc.ca/radio/spark/293-enhancing-books-wondering-where-the-time-went-and-more-1.3229541/think-a-political-party-has-flip-flopped-now-you-can-prove-it-1.3229550}{Think a political party has flip-flopped? Now you can prove it.}'' \emph{CBC Spark}, 20 September 2015. 

\ind ``\href{http://www.cbc.ca/news/canada/kitchener-waterloo/5-things-i-ve-learned-from-combing-an-archive-of-old-and-deleted-political-websites-1.3209731}{5 things I've learned from combing an archive of old and deleted political websites.}'' \emph{CBC Kitchener-Waterloo and CBC's Canada Votes portal}, analysis piece, 31 August 2015.

\ind ``\href{http://www.570news.com/2015/08/28/friday-august-28th-2015-1pm/}{Talk radio interview on WebArchives.ca}. 570 News, 28 August 2015.

\ind ``\href{http://www.cbc.ca/news/canada/kitchener-waterloo/waterloo-professor-restores-deleted-political-platforms-promises-1.3204877}{Waterloo professor restores deleted political platforms, promises,}'' \emph{CBC Kitchener-Waterloo and CBC's Canada Votes portal}, 27 August 2015. 

\ind ``\href{http://www.cbc.ca/radio/thecurrent/oct-2-2014-1.2907466/historians-want-canada-to-give-them-access-to-big-data-1.2907471}{Historians want Canada to give them access to Big Data,}'' \emph{interview on CBC's The Current with Anna Maria Tremonti}, 2 October 2014. 

%% Presentations
\marginhead{\sffamily {\vskip 0.5em}Invited Talks and Workshops}
\medskip

\ind ``Web Scraping and Digital History,'' Invited Workshop at the American Historical Association's Getting Started in Digital History event, Denver CO. January 2017.

\ind ``Making Sense of Web Archives: Facing Big Data from the Historian's Perspective,'' Invited Lecture at Library of Congress, Washington DC. June 2016.

\ind ``Historians and the Impending Big Data Deluge: How the Web Changes Everything,'' Invited Lecture at National Endowment for the Humanities, Washington DC. June 2016.

\ind (With Nick Ruest) ``Walking the WALK; Facilitating Interdisciplinary Web Archive Collaboration,'' Invited Lecture at the University of Alberta (invited by Library), Edmonton AB. June 2016.

\ind ``Archives Unleashed: Unlocking Born-Digital Sources through Interdisciplinary Collaboration,'' Keynote at Canadian Society for Digital Humanities, University of Calgary, Calgary AB. May 2016.

\ind ``Web Archives and are Reshaping Today's Historical Record,'' Invited Lecture at University of Saskatchewan Digital Humanities Talk, University of Saskatchewan, Saskatoon SK. March 2016.

\ind ``The Digital Humanities and Born-Digital Sources: Web Archives, Tweets, and the Record of Today,'' Invited Lecture at Institute for Global History Speaker Series, Georgetown University, Washington DC. February 2016.

\ind ``Big Data and History: Seeing the Past through a Macroscope,'' Invited Lecture at Selskabet for Samtidshistorisk Forskning (Danish Society for Research in Contemporary History), Copenhagen Denmark. January 2016. 

\ind ``Preserving Digital History to Save Cultural History,'' Invited Lecture at Kitchener-Waterloo Art Gallery Culture Talk, Kitchener ON. January 2016.

\ind ``Web Histories and Web Archives,'' Invited Lecture at Government Information Day, Waterloo ON. December 2015.

\ind ``An Evening with Dr. Milligan: The Challenges of Digital Sources in the Web Age: Common Tensions Across Three Web Histories, 1994-2015,'' Invited Lecture at Archives Association of Ontario Southwestern Ontario Chapter, Kitchener, ON. November 2015.

\ind ``Lost in the Infinite Archive? Web Archives for Historical Research Today... and Tomorrow,'' Keynote at Web Archives 2015: Capture, Curate, Analyze, University of Michigan, Ann Arbor MI. November 2015.

\ind ``The Challenge of Digital Sources in the Web Age: Common Tensions Across Three Web Histories, 1994-2015,'' Invited Lecture at Digital Curation Institute, University of Toronto, Toronto ON. October 2015.

\ind ``Between Content and Metadata: A Historian Faces Technical and Professional Challenges in Three Types of Web Archives,'' Invited Lecture at British Library, London UK. October 2015.

\ind ``The Challenge of Digital Sources in the Web Age: Common Tensions Across Three Web Histories, 1994-2015,'' Invited Lecture at Digital History Seminar, Institute for Historical Research, London UK. September 2015. 

\ind ``Welcome to the GeoHood: Exploring Early Web History through the GeoCities Torrent, 1994-2009,'' Invited Lecture at Centre for Internet Studies, University of Aarhus, Aarhus, Denmark. September 2015.

\ind ``Accessing Historical Data En Masse,'' Invited Workshop at Massachusetts Institute of Technology, Cambridge, MA. January 2015.

\ind ``Big Data and History: How Web Archives Will Challenge, Complement and Enhance the Historical Profession,'' Invited Lecture at Acadia Institute for Data Analytics, Acadia University, Wolfville NS. November 2014.

\ind ``Historians and the Web,'' Invited Lecture at Big Data in the SSHRC Disciplines: Data from the Past, Present and Future, Western University, London ON. March 2014.

\ind ``Historians and the Web: Challenges and Potentials of the Infinite Archive,'' University of Guelph DigiCafe Series, University of Guelph, Guelph ON. March 2014.

\ind ``Preparing for the Infinite Archive: Social Historians and the Digital Deluge,'' Keynote at Tri-University History Conference: New Approaches to History, University of Guelph, Guelph ON. March 2013.

%\end{revnumerate}

\bigskip

%\newpage

\marginhead{\sffamily {\vskip 0.5em}Conference \newline Papers}
\medskip

\ind Ian Milligan. ``Collaborative Digital History: Working with Librarians and Computer Scientists,'' American Historical Association annual meeting, Denver, Colorado. January 2017.

\ind Ian Milligan, Jimmy Lin, Jeremy Wiebe, and Alice Zhou. ``Exploring and Discovering Archive-It Collections with Warcbase,'' Digital Humanities 2016, Krakow, Poland, July 2016.

\ind Ian Milligan and Nick Ruest, ``Enabling Access to Canadian Archival Collections through WebArchives.ca,'' Canadian Society for Digital Humanities, Calgary, Alberta, June 2016. 

\ind Ian Milligan, Jimmy Lin, Jeremy Wiebe, and Alice Zhou, ``Enabling Analytics through the Warcbase Platform,'' Canadian Society for Digital Humanities, Calgary, Alberta, June 2016.

\ind Nick Ruest and Ian Milligan, ``Hands-on with Warcbase,'' International Internet Preservation Consortium Web Archiving Conference 2016, Reykjavik, Iceland, April 2016.

\ind Nick Ruest and Ian Milligan, ``An exploratory look at 3,039,804 \#elxn42 tweets'', International Internet Preservation Consortium Web Archiving Conference 2016, Reykjavik, Iceland, April 2016.

\ind Matthew Weber and Ian Milligan, ``Archives Unleashed: Hackathons as a Tool for Engaging Scholars with Web Archives,'' International Internet Preservation Consortium Web Archiving Conference 2016, Reykjavik, Iceland, April 2016.

\ind Nick Ruest and Ian Milligan, ``Enabling Access to Old Wu-Tang Clan Fan Sites: Facilitating Interdisciplinary Web Archive Collaboration,'' Code4Lib 2016, Philadelphia, Pennsylvania. March 2016. 

\ind Ian Milligan, ``Making Sense of the Web: Finding Community in the GeoCities Web Archives,'' American Historical Association annual meeting, Atlanta, Georgia. January 2016. 

\ind Ian Milligan, ``Understanding early Web history through three case studies: methodological and technical challenges,'' Times and Temporalities of the Web conference, Institut des sciences de la communication, Paris, France. December 2015. 

\ind Ian Milligan, Jimmy Lin, and Jeremy Wiebe, ``Between Metadata and Content: Exploring Canadian Political History with Archive-It's Research Services,'' Web Archives 2015: Capture, Curate, Analyze conference, University of Michigan, Ann Arbor, MI. November 2015. 

\ind Ian Milligan, ``Welcome to the GeoHood: Using the GeoCities Web Archive to Explore Virtual Communities,'' Web Archives as Scholarly Sources: Issues, Practices, and Perspectives, Aarhus, Denmark. June 2015. 

\ind Jimmy Lin and Ian Milligan, ``Warcbase: Building a Scalable Platform on HBase and Hadoop,'' Web Archiving Collaboration: New Tools and Methods, Columbia University, New York, New York. June 2015. 

\ind Ian Milligan, ``Making Sense of Abundance: Opportunity and Challenges Across Three Web Archive Case Studies,'' Canadian Society for Digital Humanities/Soci\'et\'e canadienne des humanit\'es num\'eriques (CSDH/SCHN) and Association for Computers and the Humanities (ACH) Joint Conference, University of Ottawa, Ottawa ON. June 2015. 

\ind Ian Milligan, ``The Record of Our Recent Past: Large-Scale Text Mining in Canadian Web Archives,'' Text Mining in the Humanities: Collaboration across Sectors, University of Ottawa, Ottawa ON. May 2015. 

\ind Jimmy Lin and Ian Milligan, ``Warcbase: Scaling `Out', `In', and `Down' HBase for Web Archiving,'' HBaseCon2015, San Francisco CA. May 2015. 

\ind Ian Milligan, ``WARCs, WATs, and wgets: Opportunity and Challenge for a Historian Amongst Three Types of Web Archives,'' International Preservation Committee Annual Meeting, Palo Alto CA. April 2015. 

\ind Ian Milligan, ``Web Archives and Historians: How They Can Challenge, Complement, and Shape the Historical Profession,'' Big Data, Around the World Conference, World-Wide Live Streamed Conference held at the University of Alberta. April 2015. 

\ind Ian Milligan, ``Using the GeoCities Web Archive to Explore Virtual Communities,'' Big Data in a Transdisciplinary Perspective, Hanover, Germany. March 2015. 

\ind Ian Milligan, ``The Promise of WebARChive Files: Exploring the Internet Archive as a Historical Resource,'' American Historical Association 2015 Annual Meeting, New York City, NY. January 2015.

\ind Ian Milligan, ``Clustering Search to Navigate A Case Study of the Canadian World Wide Web as a Historical Resource,'' Digital Humanities 2014 Conference, Lausanne, Switzerland. July 2014.

\ind Nick Ruest and Ian Milligan, ``The Great WARC Adventure: WARCs from Creation to Use,'' Canadian Archivists Association Annual Meeting, Victoria, BC. June 2014.

\ind Ian Milligan, ```A Haven for Perverts, Criminals, and Goons': Children and the Battle for Canadian Internet Regulation, 1991-1999,'' Canadian Historical Association Annual Meeting, Brock University, St. Catharine's ON. May 2014.

\ind Ian Milligan, ``An Infinite Archive? Historical Explorations in the Internet Archive's Wide Web Scrape,'' International Internet Preservation Committee Annual Meeting, Paris, France. May 2014. 

\ind Ian Milligan, ``Rethinking the Archival Box,'' (Digital) Humanities Revisited ? Challenges and Opportunities in the Digital Age, Hanover, Germany. December 2013.

\ind Ian Milligan, ````The Internet Archive and Social Historians: Challenge and Potential Amidst the WebARChive Files,'' Canadian Historical Association Annual Meeting, University of Victoria, Victoria BC. June 2013. 

\ind Ian Milligan, ```Active' Historians and the DIY Web?'' National Committee on Public History Annual Meeting, Ottawa ON. April 2013.

\ind Ian Milligan, ``Public History and the University: Opportunities and Challenges,'' Public History and the University Conference, York University, Toronto ON. November 2012. 

\ind Ian Milligan, ``Mining the `Internet Graveyard': Exploring Canada's Digital Collections Projects,'' Canadian Historical Association Annual Meeting, University of Waterloo and Wilfrid Laurier University, Waterloo ON. May 2012. 

\ind Ian Milligan, ``Digging into Music: An Interactive Textual Analysis of the Top 40 Billboard Lyrics Database,'' Popular Culture Association of Canada's Annual Meeting, Niagara Falls ON. May 2012. 

\ind Ian Milligan, ``Embracing the `Big History' Shift: Social Historians and Digital History (or `how I learned to stop worrying and love the n-gram'),'' Cultural Historiography: Emergent Themes and Methods, University of Guelph, Guelph ON. March 2012.

\ind Ian Milligan, ``Youth Revolt: Class and Radicalism in English Canada's Baby Boom Generation,'' Society for the History of Children and Youth Biennial Meeting, Columbia University, New York City NY. June 2011.

\ind Ian Milligan, ```The Cry of Youth': Class, Radicalism, and the Challenging of the Golden Age,'' Canadian Historical Association Annual Meeting, University of New Brunswick, Fredericton NB. May-June 2011.

\ind Ian Milligan, ``Searching for Allies: The English-Canadian New Left and Broader Community Coalitions, 1968-1973,'' 2010 Two Days of Canada Conference: `The Sixties: Canadian Style,' Brock University, St. Catharines ON. November 2010.

\ind Ian Milligan, ``Towards a New Proletariat: Debating Class and Social Change Within the English- Canadian Student New Left,'' Canadian History of Education Association Biannual Meeting, Toronto ON. October 2010.

\ind Ian Milligan, ``Growing Up on the Line: New Leftists, Youth, and Labour at the Artistic Woodwork Strike, 1973,'' Canadian Historical Association Annual Meeting, Concordia University, Montreal QC. June 2010.

\ind Ian Milligan, ``From the Quad to the Picket Line: Towards a Labour History of the Long Sixties,'' Writing the Sixties: A Practical Symposium, Carleton University, Ottawa ON. November 2009.

\ind Ian Milligan, ``No Firm Ground: The Canadian New Left and their Conceptions of Class, 1964- 1968,'' Canadian Historical Association Annual Meeting, University of British Columbia, Vancouver BC. June 2008.

\ind Ian Milligan, ``Sedition in Wartime Ontario: The Trials and Imprisonment of Isaac Bainbridge, 1918-1919,'' New Frontiers in Graduate History Conference, York University, Toronto ON. February 2008.

\ind Ian Milligan, ``Towards a New Proletariat: The Canadian New Left and the Working Class,'' Global Sixties: New World Coming Conference, Queen's University, Kingston ON. June 2007.

\bigskip 

\marginhead{\sffamily {\vskip 0.6em}Honors, \newline \& awards}
\medskip

\ind Outstanding Early Career Award, Canadian Society for Digital Humanities, 2016.

\ind Shortlisted for the John A Macdonald Prize in best Canadian historical non-fiction writing, Canadian Historical Association, 2015.

\ind Outstanding Performance Award, University of Waterloo, 2014.

\ind Best Article in the Journal of the Canadian Historical Association award, Canadian Historical Association, 2013.

\ind Avie Bennett Historica-Dominion Institute Scholarship in Canadian History, 2011-2012.

\ind SSHRC Doctoral Fellowship, 2009-2011.

\ind Ramsay Cook Fellowship for Canadian History, York University, 2009.

\ind Ontario Graduate Scholarship, Government of Ontario (declined), 2009-2010.

\ind Sir John A. Macdonald Graduate Fellowship in Canadian History, Government of Ontario. 2007-2010.

\ind Frederick W. Gibson Prize in Canadian History, Queen's University, 2006.

\ind Alexander MacLachlan Peace Prize, Queen's University, 2006.
\bigskip 

\marginhead{\sffamily {\vskip 0.6em}Grants}
\medskip

\ind \$629,576 CAD total in grant funding; \$550,147 CAD as Principal Investigator.

\medskip

\ind Microsoft Research, Microsoft Azure Research Award (``Web Archives for Longitudinal Knowledge''). Principal Investigator: \textbf{Ian Milligan}. \$20,000 USD, or \$26,804 CAD. 2016-17.

\ind Compute Canada, Research Platforms and Portals Competition (``Web Archives for Longitudinal Knowledge''). Principal Investigator: \textbf{Ian Milligan}, Co-Principal Investigator: Nick Ruest (York University). \$11,635. 2016-19.

\ind Arts and Humanities Research Council (UK) (``Born-Digital Data and Methods for History and the Humanities'').  Principal Investigator: Jane Winters (University of London), Co-Investigator: Tobias Blanke (King's College, London), Main Partners: Sonia Ranade (National Archives of the UK), Anne Alexander (University of Cambridge), James Baker (University of Sussex), \textbf{Ian Milligan} (University of Waterloo), Jason Webber and Jonathan Pledge (British Library), Peter Webster (Webster Research and Consulting). £32,210 GBP, or \$60,125 CAD. 2016-17.

\ind Social Sciences and Humanities Research Council, Connection Grant (``Unlock Your Web Archives Hackathon''), Principal Investigator: \textbf{Ian Milligan}, Co-Investigators: Jimmy Lin (University of Waterloo), Matthew Weber (Rutgers University), Nathalie Casemajor (Universit\'e du Qu\'ebec en Outaouais), Nicholas Worby (University of Toronto). \$23,715 CAD. 2015-16.

\ind Social Sciences and Humanities Research Council, Insight Grant (``A Longitudinal Analysis of the Canadian World Wide Web as a Historical Resource, 1996-2014''). Principal Investigator: \textbf{Ian Milligan}, Co-investigators: Nick Ruest (York University), William Turkel (Western University). \$257,541 CAD. 2015-20.

\ind Ontario Ministry of Research and Innovation, Early Researcher Award (``A Longitudinal Analysis of the Canadian World Wide Web as a Historical Resource, 1996---2014''). \$150,000 CAD. Principal Investigator: \textbf{Ian Milligan}. \$150,000 CAD. 2015-20.

\ind Social Sciences and Humanities Research Council, Connection Grant (``New Directions in Active History: Institutions, Communication, and Technologies''). Principal Investigator: Thomas Peace (Huron University College), Co-Investigators: \textbf{Ian Milligan}, Jim Clifford (University of Saskatchewan), Daniel Ross (York University), Kaleigh Bradley (York University), Krista McCracken (Algoma University), Andrew Nurse (Mount Allison University). \$19,304 CAD. 2015-16.

\ind Social Sciences and Humanities Research Council, Insight Development Grant (``An Infinite Archive? Developing HistoryCrawler to Explore the Internet Archive as a Historical Resource''). Principal Investigator: \textbf{Ian Milligan}, Co-Investigator: William Turkel (Western University).  \$74,952 CAD. 2013-15.

\ind University of Waterloo/Social Sciences and Humanities Research Council Institutional Block Grant (SEED Grant for same project as above), June 2013 --- June 2015. Principal Investigator: \textbf{Ian Milligan}. \$5,500 CAD. 2013-15.
\medskip 

%% Graduate Teaching

\marginhead{\sffamily {\vskip 0.35em}Teaching}
\medskip
\noindent\emph{Doctoral Training \vspace{0.01in}}

\ind PhD Supervisor for Sarah McTavish, 2016-Present.

\ind Doctoral Committee for Victoria Campbell, 2013-Present.

\ind External Examiner for Kimberly Martin, Faculty of Information \& Media Studies, University of Western Ontario. 2016.

\ind Internal/External Examiner for Jason Hawreliak, Department of English. 2013.

\medskip
\end{document}